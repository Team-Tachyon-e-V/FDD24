\subsection{Overview}
%Electrical overview content here
(a) Explain the main requirements and constraints that drive the design. \\
\par Our general design of this season is inspired by conventional modes of transportation,
as we have not had the capacity to start developing a levitation system by this season
Thus, we focussed on an electrical system that drives our friction-based motor with
excelling acceleration, which is a problem that railway systems frequently face.

\par In between design and production phase, we received a sponsorship of Leadrive,
a local startup for research on automotive power electronics.
Therefore, our workload was eased, which turned out to be favorable
because of our lack of team members in the electrical field. This has been a crucial
constraint in the design and planning process of the electrical department
since the last season. Only shortly before submitting the ITD, we were able to make an estimation of
realistic goals for the new team.

\par This year, we would like to set the path for magnetic levitation in the future, relying on an active system
inside the vehicles. This was taken into consideration when designing the power dimensions,
keeping plenty of overhead for the future, which aligns with our goal of sustainability.
By having reusable modules, the design process of the upcoming years will be simplified.

\par Our project has been significantly bolstered by the generous support and endorsement from leading industry giants, which has considerably alleviated our financial burdens and propelled our initiative towards groundbreaking achievements.

Altium has been instrumental, providing us with cutting-edge PCB design software valued at approximately \euro 10,000. This invaluable resource has empowered our team to design highly complex and efficient circuit boards, essential for the intricate electronics that drive our hyperloop prototype.

Festo, renowned for their pneumatic and automation solutions, contributed a suite of components and systems worth over \euro 15,000. Their support has enhanced our prototype's propulsion and control systems, enabling precise maneuverability and stability at high velocities.

Mouser Electronics stepped in with a crucial contribution, supplying us with electronic components and parts worth about \euro 20,000. This vast array of high-quality components has been pivotal in assembling our prototype's electrical systems, ensuring reliability and performance.

Würth Elektronik provided essential PCB materials and expertise, along with a donation of specialized components valued at \euro 12,000. Their contributions have significantly optimized our prototype's power distribution and structural integrity.

Leadrive offered their advanced inverter technology, a contribution that not only included hardware valued at \euro 18,000 but also critical technical support. This has dramatically improved our prototype's efficiency and power management capabilities.

Vector Informatik has generously supplied software licenses and technical support for vehicle communication systems, with a contribution valued at \euro 8,000. This support has been crucial in implementing robust and reliable communication systems within our prototype.


\subsection{Electrical and mechanical design process}
%Electrical and mechanical design process content here
\subsubsection{(a) Present Schematics or logic diagrams of the boards.}
\subsubsection{(b) Present temperature simulations for vacuum conditions.}
For our heat simulations, we used the software of ANSYS. By vacuum conditions, we assumed the
lack of gas flow, which eliminates the cooling heat flow from winds. The simulation tool solves
the heat transfer equation \( \frac{\partial T}{\partial t} = \alpha \left( \frac{\partial^2 T}{\partial x^2} + \frac{\partial^2 T}{\partial y^2} + \frac{\partial^2 T}{\partial z^2} \right) \)
by discretizing through Finite-Element-Methods.

\subsection{Description of subsystem control}
%Description of subsystem control content here
\subsubsection{(a) Briefly reference the control systems of the boards, which should be explained in the levitation or propulsion subsection respectively.}
We configure the BMS prior to the competition. \\


\subsection{Electrical system characteristics}
%Content of the electrical system characteristics here
\begin{table}
    \centering
    \begin{adjustbox}{width=\textwidth,center}
    \begin{tabular}{|c|c|}
       \hline
       \textbf{Column 1} & \textbf{Column 2}  \\
       \hline
       Battery Type & Lead-Acid(integrated)\\
       \hline
       Capacity[Ah] & 9 \\
       \hline
       Normial Voltage[V] & 12 \\
       \hline
       Cell configuration & 2s \\
       \hline
       Maximum discharge current (lower limit) [A] & 10 \\
       \hline
       Weight per cell [Kg] & 2,7 \\
       \hline 
       Dimensions per cell (L x W x H)[mm] & 151 x 65 x 94 \\
       \hline 
    \end{tabular}
    \end{adjustbox}
    \label{Low Voltage Battery Specs}
\end{table}    

\begin{table}[H]
    \centering
    \begin{adjustbox}{width=\textwidth,center}
    \begin{tabular}{|l|c|c|c|l|l|}
        \hline
        \textbf{Parameter} & \textbf{MIN} & \textbf{NOM} & \textbf{MAX} & \textbf{Unit} & \textbf{Conditions} \\
        \hline
        Ambient Temp. for Operation ($T_{\text{AMB}}$) & -40 & - & 90 & °C & - \\
        \hline
        Ambient Temp. for Storage ($T_{\text{STO}}$) & -40 & - & 85 & °C & - \\
        \hline
        Relative Humidity & 0 & - & 95 & \% & - \\
        \hline
        Flow Rate of Coolant ($V_{\text{CLNT}}$) & 8 & 12 & 16 & l/min & Derating @ $8\sim12$ l/min \\
        \hline
        Inlet Temp. of Coolant ($T_{\text{CLNT}}$) & -40 & - & 85 & °C & Derating @ $65\sim85$°C \\
        \hline
        Cooling Inlet Pressure ($P_{\text{INLET}}$) & - & - & 2.5 & bar & - \\
        \hline
        Pressure Drop between Cooling Inlet and Outlet ($P_{\text{DROP}}$) & - & 0.25 & - & bar & $T_{\text{CLNT}}=65°C$, $V_{\text{CLNT}}=12$ l/min \\
        \hline
        Input Voltage ($V_{\text{DC}}$) & 260 & 600 & 850 & V & Full operation @ $450-800$V \\
        \hline
        Input Current ($I_{\text{DC}}$) & - & 200 & - & A & Continuous \\
        \hline
        Peak Input Current ($I_{\text{DCPK}}$) & - & 300 & - & A & For max $t_{\text{PK}}$ duration \\
        \hline
        Output Voltage ($V_{\text{AC}}$) & - & 400 & - & Vrms & - \\
        \hline
        Output Current ($I_{\text{AC}}$) & - & - & 200 & Arms & Continuous \\
        \hline
        Peak Output Current ($I_{\text{ACPK}}$) & - & - & 300 & Arms & For max $t_{\text{PK}}$ duration \\
        \hline
        Output Power ($S_{\text{AC}}$) & - & 135 & - & kVA & Continuous \\
        \hline
        Peak Output Power ($S_{\text{ACPK}}$) & - & 200 & - & - & For max $t_{\text{PK}}$ duration \\
        \hline
        Peak Duration ($t_{\text{PK}}$) & - & - & 60 & s & - \\
        \hline
        Input Voltage for Control ($V_{\text{BAT}}$) & 6 & - & 36 & V & Full functional @ $8-32$V (control board) \\
        \hline
        Max. Efficiency ($\eta$) & 97 & - & - & \% & - \\
        \hline
        Torque Control Accuracy ($\epsilon_{\text{TRQ}}$) & - & - & 3 & \% & Torque $>100$Nm \\
        \hline
         & - & - & 3 & Nm & Torque $<100$Nm \\
        \hline
        Torque Control Speed ($t_{\text{TRQ}}$) & - & - & 100 & ms & - \\
        \hline
        Speed Control Accuracy ($\epsilon_{\text{SPD}}$) & - & - & 30 & rpm & - \\
        \hline
    \end{tabular}
    \end{adjustbox}

    \caption{800V Single Inverter Specifications}
    \label{inverter_specs}
\end{table}

\subsection{Interface with other system}
%Interface with other system content here n
(a) Briefly reference the communication protocols or control mechanisms of the boards, which should be explained in the respective Sense and Control subsection. \\
All the electric subsystems are located within the pod. \\

The phsyical connection matrix is as following:
\begin{table}
    \centering
    \begin{adjustbox}{width=\textwidth,center}
    \begin{tabular}{|c|c|c|c|c|c|c|}
    \hline
    From | To & \text{LV Battery} & \text{HV Battery} & \text{BMS} & \text{Traction Inverter} & \text{Motor} & \text{Cooling System} \\
    \hline
    \text{LV Battery} & - & - & Powers & \text{Powers control system} & - & \text{Powers pump and control system} \\
    \hline
    \text{HV Battery} & - & - & Connects to & \text{Provides power} & - & - \\
    \hline
    \text{BMS} & - & - & \text{Controls} & - & - & - \\
    \text{Traction Inverter} & - & - & - & - & \text{Propels} & X \\
    \hline
    \text{Motor} & - & - & - & - & - & - \\
    \hline
    \text{Cooling System} & - & - & - & \text{Cooling} & \text{Cooling} & \text{Cooling (implicitly)} \\
    \hline
    \end{tabular}
\end{adjustbox}
\caption{Physical connection matrix}
\label{Physical connection matrix}
\end{table}

The data connection matrix is as following. All communication between boards are via CAN, if not specified otherwise:

\begin{table}
    \centering
    \begin{adjustbox}{width=\textwidth,center}
    \begin{tabular}{|l|c|c|c|c|c|c|c|c|}
    \hline
    From $\backslash$ To & LV Battery & HV Battery & BMS & Traction Inverter & Motor & Cooling System & Brakes Controller & Telemetry Unit  \\
    \hline
    LV Battery & - & - & - & - & - & - & - & - \\
    HV Battery  & - & - & Discharge rate, voltage level & - & - & - & - & - \\
    BMS & controls & controls & - & - & - & - & - & sends data \\
    Traction Inverter & - & - & - & - & - & - & - & sends data \\
    Motor & - & - & - & - & - & - & - & - \\
    Cooling System & - & - & - & - & - & - & - & sends data \\
    Brakes Controller & - & - & - & - & - & - & - & sends data \\
    Telemetry Unit & - & - & updates limits & sends commands & - & sends target rates & sends commands & - \\
    \hline
    \end{tabular}
    \end{adjustbox}
    \caption{Data connection matrix}
    \label{data-connectivity-matrix}
\end{table}


\subsection{Final system description}


\subsection{Manufacturing process}
%Manufacturing process content here
Our PCB Design
\subsubsection{PCBs}
\par Prototyping: Prototype PCBs are fabricated in the FabLab associated with our university. The FabLab provides access to PCB manufacturing equipment and materials, enabling the rapid production of prototypes for initial testing and design validation.
    Once the PCBs are fabricated, they are assembled manually by our team members. 
\par Production: We ordered our final PCBs from JLCPCB, a leading PCB manufacturing service. In addition to JLCPCB, we also collaborate with Würth Elektronik who produce 
PCBs in Germany, aligning with our goal of sustainability.

\subsubsection{Inverter}
The inverter is a product from Leadrive used as an OEM product in automotive and mobility industry. \\

    Support from Leadrive: The development of the inverter system is supported by Leadrive, a company specializing in advanced inverter technology. Their expertise significantly contributes to the optimization of our propulsion system.
    Collaboration with Formula Student Team of FH Aachen: Additionally, we collaborate with the Formula Student Team of FH Aachen, benefiting from their practical experience in electric vehicle design and inverter application. This partnership enriches our project with valuable insights into inverter integration and performance enhancement.

\subsection{Testing}
%Testing content here
We started testing software.

%\subsection{FMEA}
%FMEA content here
