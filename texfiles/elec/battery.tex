\subsection{Overview}
%Electrical overview content here
(a) Explain the main requirements and constraints that drive the design. \\


\subsection{Electrical and mechanical design process}
%Electrical and mechanical design process content here
\subsubsection{(a) Present Schematics or logic diagrams of the boards.}
\subsubsection{(b) Present temperature simulations for vacuum conditions.}
For our heat simulations, we used the software of ANSYS. By vacuum conditions, we assumed the
lack of gas flow, which eliminates the cooling heat flow from winds. The simulation tool solves
the heat transfer equation \( \frac{\partial T}{\partial t} = \alpha \left( \frac{\partial^2 T}{\partial x^2} + \frac{\partial^2 T}{\partial y^2} + \frac{\partial^2 T}{\partial z^2} \right) \)
by discretizing through Finite-Element-Methods.

\subsection{Description of subsystem control}
%Description of subsystem control content here
\subsubsection{(a) Briefly reference the control systems of the boards, which should be explained in the levitation or propulsion subsection respectively.}
We configure the BMS prior to the competition. \\


\subsection{Electrical system characteristics}
%Content of the electrical system characteristics here
   


\subsection{Interface with other system}
%Interface with other system content here n
(a) Briefly reference the communication protocols or control mechanisms of the boards, which should be explained in the respective Sense and Control subsection. \\
All the electric subsystems are located within the pod. \\

The phsyical connection matrix is as following:
\begin{table}
    \centering
    \begin{adjustbox}{width=\textwidth,center}
    \begin{tabular}{|c|c|c|c|c|c|c|}
    \hline
    From | To & \text{LV Battery} & \text{HV Battery} & \text{BMS} & \text{Traction Inverter} & \text{Motor} & \text{Cooling System} \\
    \hline
    \text{LV Battery} & - & - & Powers & \text{Powers control system} & - & \text{Powers pump and control system} \\
    \hline
    \text{HV Battery} & - & - & Connects to & \text{Provides power} & - & - \\
    \hline
    \text{BMS} & - & - & \text{Controls} & - & - & - \\
    \text{Traction Inverter} & - & - & - & - & \text{Propels} & X \\
    \hline
    \text{Motor} & - & - & - & - & - & - \\
    \hline
    \text{Cooling System} & - & - & - & \text{Cooling} & \text{Cooling} & \text{Cooling (implicitly)} \\
    \hline
    \end{tabular}
\end{adjustbox}
\caption{Physical connection matrix}
\label{Physical connection matrix}
\end{table}

The data connection matrix is as following. All communication between boards are via CAN, if not specified otherwise:

\begin{table}
    \centering
    \begin{adjustbox}{width=\textwidth,center}
    \begin{tabular}{|l|c|c|c|c|c|c|c|c|}
    \hline
    From $\backslash$ To & LV Battery & HV Battery & BMS & Traction Inverter & Motor & Cooling System & Brakes Controller & Telemetry Unit  \\
    \hline
    LV Battery & - & - & - & - & - & - & - & - \\
    HV Battery  & - & - & Discharge rate, voltage level & - & - & - & - & - \\
    BMS & controls & controls & - & - & - & - & - & sends data \\
    Traction Inverter & - & - & - & - & - & - & - & sends data \\
    Motor & - & - & - & - & - & - & - & - \\
    Cooling System & - & - & - & - & - & - & - & sends data \\
    Brakes Controller & - & - & - & - & - & - & - & sends data \\
    Telemetry Unit & - & - & updates limits & sends commands & - & sends target rates & sends commands & - \\
    \hline
    \end{tabular}
    \end{adjustbox}
    \caption{Data connection matrix}
    \label{data-connectivity-matrix}
\end{table}


\subsection{Final system description}
%Description of the system here
\subsubsection{Battery Cells}

High Voltage Network:

Our high voltage battery will make use of lithium-ion polymer technology. We use 120 pouch-format cells from Shenzhen GrePow Battery Co. Ltd rated at 45C maximum discharge that we plan to connect in series. 
The finished package (main battery pack) will be assembled by the team.
We are going to connect the 120 cells connected in series and that will have 1 parallel line. This will roughly have 504 Volt at max (using \(120 * 4.2V = 504 V \) ) 
which provides sufficient electricity to power the motor.
The battery pack will provide up to ~350 Amps of DC current available to the inverter. However, neither the inverter nor the motor is not rated for such high currents nominally. 
\newline
Therefore, the maximum output current of the HV Battery will be rated at 200 A maximum (peak) and 100 A continuous. 

We will stack 30 cells in series per pack and then stack 4 of them to get the full battery pack. No we won't.
\newline
\subsubsection{BMS}
Our battery management system 

The Orion BMS 2, connected to the HV battery, protects it and improves its life, efficiency. 
Operational Mechanics

The Orion BMS 2 facilitates real-time monitoring and management of each cell within the HV battery pack, 
which consists of 120 lithium-ion polymer cells arranged in a series configuration to achieve a nominal voltage of 504V. 
This arrangement necessitates precise control and monitoring to prevent overcharging, deep discharging, and to ensure balanced 
cell voltages, all of which are within the Orion BMS 2's capabilities.

1. **Cell Voltage Monitoring and Balancing:** The BMS continuously monitors the voltage of each cell, ensuring 
that all cells operate within their safe voltage range. Cell balancing is performed to equalize the charge across all cells,
thereby enhancing the battery pack's overall efficiency and lifespan.

2. **Temperature Monitoring:** 
Given the high energy density of the HV battery pack, 
thermal management is paramount. The Orion BMS 2 monitors the temperature of individual cells 
and the battery pack as a whole, activating cooling measures when necessary and preventing operation 
under extreme temperatures that could damage the battery or compromise safety.
The Orion BMS 2 itself tracks the temperature of 8 individual cells. 28 other cells are measured by the Thermal Controller.

3. **State of Charge (SoC) and State of Health (SoH) Estimation:** 
SoC and SoH estimations are important for optimal battery utilization and health maintenance. 
The Orion BMS 2 employs advanced algorithms to provide these estimates, 
ensuring that the battery's capacity is used efficiently.


The integration of the Orion BMS 2 encompasses several safety mechanisms designed to protect the battery pack, the hyperloop pod, and its occupants:

1. **Overcurrent and Short Circuit Protection:** By monitoring the current flowing in and out of the battery pack, the Orion BMS 2 can detect overcurrent conditions and short circuits, initiating immediate shutdown procedures to prevent damage and ensure safety.

2. **High and Low Voltage Protection:** The BMS prevents the battery from exceeding its maximum voltage during charging and dropping below its minimum voltage during discharge, thereby avoiding scenarios that could lead to reduced battery life or safety hazards.

3. **Thermal Runaway Prevention:** Through its temperature monitoring capabilities, the Orion BMS 2 can detect the onset of thermal runaway—a dangerous condition where one cell's failure can lead to a cascading failure of adjacent cells—and take corrective actions to isolate the problem and mitigate potential damage.

Efficiency Enhancements

By optimizing the operational parameters of the HV battery pack, the Orion BMS 2 contributes significantly to the efficiency and performance of the hyperloop prototype:

1. **Energy Optimization:** By ensuring that all cells are balanced and operate within their optimal voltage and temperature ranges, the BMS maximizes the energy extracted from the battery pack, contributing to the hyperloop's range and speed capabilities.

2. **Lifecycle Extension:** Through diligent monitoring and management, the Orion BMS 2 extends the useful life of the HV battery pack, reducing the environmental impact and operational costs associated with battery replacement.

3. **Predictive Maintenance:** By providing detailed data on the SoC and SoH, the Orion BMS 2 enables predictive maintenance, allowing for timely interventions that prevent unscheduled downtimes and extend the battery's lifespan.

Conclusion

The integration of the Orion BMS 2 with the HV battery pack in our hyperloop prototype represents a critical step towards ensuring the system's safety, efficiency, and reliability. Through its comprehensive monitoring and management capabilities, the Orion BMS 2 ensures that the HV battery pack operates within its optimal parameters, significantly contributing to the prototype's overall performance and safety profile. As we progress towards the final stages of the FDD, the detailed exploration of the Orion BMS 2's functionalities underscores our commitment to leveraging advanced technologies for the enhancement of hyperloop transportation systems.
\subsubsection{Insulation Monitoring Device}
The Insulation Monitoring Device that is mandatory for EHW participants, as well as Formula Students teams, is not built inhouse, after receiving the respective advice from the EHW technical jury. By reaching out to Bender, we received their device through their Formula Students policy. It is configured for ...


\subsection{Manufacturing process}
%Manufacturing process content here
Our PCB Design
\subsubsection{PCBs}
\par Prototyping: Prototype PCBs are fabricated in the FabLab associated with our university. The FabLab provides access to PCB manufacturing equipment and materials, enabling the rapid production of prototypes for initial testing and design validation.
    Once the PCBs are fabricated, they are assembled manually by our team members. Bigger PCBs are assembled in the facilities of the FabLab with the manual Pick and Place Machine and a reflow oven.
\par Production: We ordered our final PCBs from JLCPCB, a leading PCB manufacturing service. In addition to JLCPCB, we also collaborate with Würth Elektronik who produce 
PCBs in Germany, aligning with our goal of sustainability.
\subsubsection{Batteries}
The production of the low voltage battery pack is rather easy. We will use a 3d printer to print the casing
We produced the battery packs in cooperation with the ISEA (Institute for Power Electronics and Electrical Drives) at RWTH, whose experience helped us to assemble and design the parts
more efficienciently and more safely, as we had a considerable high voltage system.
The casing will consist of polycarbonate. Polycarbonate is a material that is
durable , lightweight, and impact resistant. The problems with
material degradation through UV emissions does not impact us substantially, as we
cover the battery pack inside the shell for most of the time. We will have safety measures preventing too much exposure to UV radiation.
Also, the degradation is mainly of cosmetic nature (https://link.springer.com/article/10.1007/s11668-020-01002-9) .

\subsection{Testing}
%Testing content here
We started testing software.

%\subsection{FMEA}
%FMEA content here
