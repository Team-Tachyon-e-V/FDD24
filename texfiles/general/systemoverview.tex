\chapter{Technical System Overview}

Scaling for a full scale model:
A full-scale model of our hyperloop is theoretically not unfeasible. Our gauge of 900mm is frequently used in freight rails. Electrically, our whole system has a lot of headroom, and many components, such as the motor or the traction inverter, come from bigger applications (electric cars, aviation). Thus, this would not pose a problem. 
Add: Pressure conditions, temperature conditions, security, sense+control, track switching.

\subsection{Transport}
transport by car.

\begin{table}[htbp]
\centering
\caption{System Overview}
\begin{tabular}{|l|l|}
\hline
Length [m] & 3 \\
Width [m] & 1 \\
Height [m] & 0.5 \\
Weight [kg] & 200 \\
Vertical levitation method & None \\
Motor concept & friction-based PMSM \\
Motor max. current [A] & 120 \\
Regenerative braking used & No \\
LV system max. voltage [V] & 26 \\
HV system max. voltage [V] & 504 \\
Braking method & Pneumatic activated friction pads \\
Pneumatic system max. pressure [bar] & 9 \\
Data Log Frequency [Hz] & 10-100 \\
Board communication protocols & CAN \\
\hline
\end{tabular}
\caption{System overview}
\label{tab: system-overview}
\end{table}
