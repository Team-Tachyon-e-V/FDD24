\chapter{Introduction}

Tachyon Hyperloop e.V. is a pioneering initiative conceived in 2019 by students from RWTH Aachen University and FH Aachen University of Applied Sciences. Inspired by Elon Musk's groundbreaking vision of the Hyperloop in 2013, our team comprises 35 dedicated students from diverse academic backgrounds, all unified by our shared commitment to pushing the boundaries of knowledge and technology within the transportation sector.

Our team functions seamlessly across six specialized departments: Mechanical, Electrical, Project Management, Business and Marketing, Sponsoring, and Architecture \& Urban Design. Each department is led by a proficient team lead, ensuring clear communication and effective coordination across all project facets. Guided by the strategic oversight of board members including Jacob Diercks, Marijan Schlösser, and Yashasvi Karnena, we navigate the complex landscape of Hyperloop innovation with determination and vision.
\newpage
\section{Applicant and List of Team Members}
\begin{table}[H]
\centering
\begin{minipage}{0.45\textwidth}
\centering
\begin{tabular}{|l|l|}
\hline
\textbf{Name}           & \textbf{Department}        \\ \hline
Jacob Diercks            & First Chairman             \\ \hline
Marijan Schlösser       & Second Chairman            \\ \hline
Yashasvi Karnena        & Treasurer                  \\ \hline
Pascal Pfeifer          & Mechanical Department      \\ \hline
Lennart Jepsen          & Mechanical Department      \\ \hline
Shreepad Khedkar        & Mechanical Department      \\ \hline
Parthik --------        & Mechanical Department      \\ \hline
Benjamin Köhler         & Mechanical Department      \\ \hline
Jasmin Dedeoglu         & Mechanical Department      \\ \hline
Aniket Saxena           & Mechanical Department      \\ \hline
Kanishk Singh           & Mechanical Department      \\ \hline
Yash Shah               & Mechanical Department      \\ \hline
Rengin Solmaz           & Mechanical Department      \\ \hline
Kanishk Singh         & Mechanical Department      \\ \hline
Hardik -------             & Mechanical Department      \\ \hline
Aryan Modi              & Mechanical Department      \\ \hline
\end{tabular}
\caption{Team Members and Departments}
\label{tab:left}
\end{minipage}
\hfill
\begin{minipage}{0.45\textwidth}
\centering
\begin{tabular}{|l|l|}
\hline
\textbf{Name}           & \textbf{Department}        \\ \hline
Dino Cheng              & Electrical Department      \\ \hline
Abhishek Jha            & Electrical Department      \\ \hline
Sourajit Majumder       & Electrical Department      \\ \hline
Stefan Boskovic         & Electrical Department      \\ \hline
Bohdan Popov            & Electrical Department      \\ \hline
Ian Morales             & Electrical Department      \\ \hline
Paula Vicente           & Electrical Department      \\ \hline
Arya Kumalajati        & Electrical Department      \\ \hline
Praneet Tuli            & Project Management         \\ \hline
Duc Thiem Do            & Project Management         \\ \hline
Till Behringer          & Project Management         \\ \hline
Ludwig Gatzsch          & Business \& Marketing      \\ \hline
Henriette Brucks        & Business \& Marketing      \\ \hline
Joana Baumann           & Sponsoring                 \\ \hline
Carl-Philipp Schetelig  & Sponsoring                 \\ \hline
Clara Kamrath           & Architecture \& Urban Design \\ \hline
Hannes Wittkopf         & Architecture \& Urban Design \\ \hline
Janik Hiob              & Architecture \& Urban Design \\ \hline
\end{tabular}
\label{tab:right}
\end{minipage}
\end{table}


\section{Development Environment and Research Objectives}
Despite the demands of our academic pursuits, we dedicate ourselves fully to the Tachyon initiative. From our inception, we have found a home at the Collective Incubator, a vibrant hub that fosters collaboration and innovation. Previously, we benefited from the facilities at the WZL, an institute of RWTH Aachen University, before transitioning to our current workspace. This dynamic environment has nurtured our growth and facilitated the realization of our ambitious projects.

Our endeavors are sustained through a blend of financial support from sponsors, generous donations, and contributions from supporting organizations. Tachyon Hyperloop epitomizes our collective aspiration to design and construct functional prototypes of Hyperloop Pods, contributing significantly to the advancement of transportation technology on a global scale.

Acknowledging our sponsors is pivotal to the success of our projects. We extend our heartfelt appreciation to RWTH Aachen University for their unwavering support throughout our journey. Special gratitude is also extended to Institut für Allgemeine Mechanik for their invaluable expertise and infrastructure. Additionally, we express our gratitude to proRWTH, WZL, and International Academy for their unwavering support. We extend our heartfelt thanks to EVS Euregio GmbH for providing a testing track and to Collective Incubator e.V. for their provision of an exceptional office space. Special thanks are extended to our sponsor-partners STAWAG, Emrax, Mouser, Contitech AG, Leaddrive, Vector, PWC and Dürr  for their substantial material and financial support. Finally, we would like to express our appreciation to the European Hyperloop Week Committee for providing us with the opportunity to showcase our project, facilitating collaboration, information sharing, and networking within the Hyperloop community.

Apart from our primary projects highlighted in this documentation, Tachyon Hyperloop e.V. is also engaged in other initiatives that are integral to our research and development.

One such project is Pathfinder. As one of the few teams to have our own test track near Aachen, generously provided by our partner EVS EUREGIO Verkehrsschienennetz GmbH, our 1 km long test track stands as one of the longest among student teams. For the upcoming European Hyperloop Week 2024 competition, we have collaborated with Swissloop to utilize their track infrastructure. This partnership aims to foster knowledge exchange, allowing us to gain valuable insights into the limitations and advantages of their track design. By sharing the same tracks, we also anticipate significant cost efficiencies. The strategic collaboration between our teams promises mutual benefits, facilitating innovation and advancement within the Hyperloop community.

Additionally, the Minipod project plays a vital role in our development process. Serving as a prototype of the prototype, the Minipod enables us to mitigate risks associated with resource allocation and system functionality before implementing them on a larger scale. By developing and testing on a smaller scale, we can gather valuable insights and refine our technologies with reduced risk.

In the upcoming sections, we will delve into a comprehensive System Overview, offering a succinct understanding of the Pod. Following this, we will explore the Mechanical system, intricately detailing the design and construction of our pod. Our focus will then shift to the Electrical domain, where we will discuss vital components such as the Sense \& Control systems crucial for our pod's functionality. Additionally, we will address Safety measures implemented to ensure the well-being of our team and the success of of our project. Through these sections, our aim is to provide a detailed insight into our project's development process, while adhering to the guidelines and regulations outlined by the European Hyperloop Week Committee.

\section{Category for This Application}