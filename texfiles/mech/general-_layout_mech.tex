\subsection{Overview}
\subsubsection{Requirements and Constraints}
.  Detail the main requirements and constraints driving the design.
\subsubsection{Concept}
.  Explain the concept of the subsystem and reasons for its selection.
\subsubsection{Size, Components, and Appearance}
.  Include a table of materials, mass, dimensions, and other relevant factors.

\subsection{Design Process and Appearance}
\subsubsection{Requirements Met by the Design}
.  Outline different requirements to be met by the infrastructure.
\subsubsection{Design Rationale}
.  Detail the design rationale behind the components of the infrastructure.
\subsubsection{CAD Models and Technical Drawings}
.  Present CAD models and technical drawings.
\subsubsection{FEM Results}
.  Present FEM results for worst-case scenarios, including images and values.
\subsubsection{Mesh and Boundary Conditions}
.  Provide details on the type of mesh, boundary conditions, and Free Body Diagrams.
\subsubsection{Infrastructure Challenges}
.  Address thermal expansion, weather resistance, etc.

\subsection{Manufacturing Process}
.  Compile a parts list in tabular format, specifying in-house or outsourced production.

\subsection{Transport and Assembly Process}
\subsubsection{Transport and Lift Plan}
.  Describe the transport and lift plan.
\subsubsection{Assembly Process}
.  Describe the assembly process, including integration into subordinate systems.
\subsubsection{Timeline and Equipment}
.  Include a timeline, equipment needed, and workforce details.

\subsection{Safety Considerations}
\subsubsection{Safety Factor}
.  Discuss the safety factor applied.
\subsubsection{Worst-Case Scenarios}
.  Discuss worst-case scenarios and containment plans.
\subsubsection{Transport, Storage, and Lifting Requirements}
.  Requirements as per Section 9.3, especially TS.4. of the R&R.
\subsubsection{Physical Stop and Roll-Over Calculations}
.  Describe the physical stop and roll-over calculations.

\subsection{FMEA Results Discussion}
\subsubsection{Risk Assessment}
.  Preliminary risk assessment for demonstration, transport, and lifting.
\subsubsection{FMEA and Risk Mitigation}
.  Detail FMEA and describe risk mitigation measures.
\subsubsection{Simulation Evidence}
.  Provide evidence of simulations validating theoretical assumptions.

\subsection{Testing}
.  Describe testing procedures and provide a preliminary testing plan.

\subsection{Demonstration}
.  Outline how the system will be demonstrated if standalone.